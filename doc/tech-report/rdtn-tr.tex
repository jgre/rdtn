\documentclass{article}

\usepackage[latin1]{inputenc}
\usepackage{graphicx}
\usepackage{times}

\bibliographystyle{plain}

\begin{document}

\title{RDTN: A DTN Bundle Protcol Agent implemented in Ruby}

\author{Janico Greifenberg \and Dirk Kutscher}
\date{$Date$\\
$Rev$}

\maketitle

\begin{abstract}

RDTN is a DTN bundle protocol agent (BPA) implementation written in Ruby. RDTN
is light-weight and flexible so that it can be used in DTN application and
protocol development as well as in research of DTN-related topics such as
routing or convergence-layers. 

RDTN provides a language-independent interface for client application, an
interactive environment for tests and a simulation mode for running multiple
instances of the RDTN code in a simulated network environment. RDTN implements
the bundle protocol (RFC 5050), convergence-layer adapters for TCP, UDP, and
FLUTE, and static and epidemic routing as well as the DTN Publish/Subscribe
Protocol (DPSP).

\end{abstract}

\section{Introduction}\label{sec.intro}

\section{Architecture}\label{sec.arch}

EventDispatcher

Threads

\section{Convergence Layer Adapters}\label{sec.cl}

\section{Contact Management}\label{sec.contact-mngt}

\section{Routing}\label{sec.routing}

\section{Persistent Storage}\label{sec.storage}

\section{Application Interface}\label{sec.appif}

The RDTN application interface allows applications to send and receive bundles,
query the persistent storage, and send application acknowledgements. This
interface can be accessed inside the RDTN daemon process, so that it can be used
by the interactive mode \ref{sec.interactive} and by extensions
\ref{sec.extending}, as well as from separate processes either on the same
machine or on another one connected via a network.

\subsection{Bundle Objects}\label{sec.bundle-obj}

Applications can use the internal representation of bundles directly so that
they can manipulate all fields, flags, and extension blocks. Opening the
internal representation to applications lets us avoid the overhead of mirroring
functionality and interfaces from internal to public interfaces and it gives
applications the ability to use parts of the bundle (e.g. extensions) that are
not explicitly handled by RDTN.

However, the small overhead and high flexibility of using the internal bundle
objects comes with the risk of complicating application development by forcing
developers to deal with an abundance of options and corner cases unnecessary for
typical use cases. Incorrect use of this interface which allows access to
fundamental parts of the protocol implementation may also cause invalid messages
to be sent over the network.

The complexity of the interface is mitigated by using default values covering the
typical use cases. Thus, application developers do not need to know the details
of the bundle protocol to use the API. But it is still possible to leverage
knowledge about the protocol and possibly extensions to implement more advanced
functionality.

RDTN bundle objects are implemented by {\tt Bundling::Bundle}. The objects
contain a list of Bundle blocks, the first of which is always the primary bundle
block (implemented by {\tt Bundling::PrimaryBundleBlock}). The list of blocks
also contains an arbitrary number of objects derived from {\tt Bundling::Block},
which includes the payload block ({\tt Bundling::PayloadBlock}). Bundle objects
have a flag indicating whether custody was accepted for the bundle, a list of
custody timers {\bf somewhat incorrect now}, and a log of the forwarding 
operations of this bundle. The
log is implemented by {\tt Bundling::ForwardLog}, storing an entry for
each time the bundle is received, transmitted, or a transmission error for the
bundle is detected. An entry comprises of a timestamp, the action (i.e. if the
bundle was received, replicated, or forwarded), the status (i.e. transmitted,
in-flight, or transmission error), and a reference to
the neighbor involved in the operation (i.e. the receiver or the sender
depending on the action).

{\bf TBD: the stuff about the forwardlog does not belong in this section}

Bundle objects can be created optionally passing payload, source EID,
destination EID, report-to EID, and custodian EID as parameters. Although the
bundle class does not save most of the values such as source, destination, and
payload itself but keeps them in the block classes, a bundle object responds
directly to function calls by delegating them to the primary bundle block object
(in the case of all fields and flags specified for that block in RFC 5050
\cite{bundle-spec}) or to the payload block. Other blocks can be accessed
through the list of blocks ({\tt Bundling::Bundle\#blocks}).

\subsection{Sending Bundles}\label{sec.sending}

RDTN provides two functions for sending bundles: {\tt sendBundle} and {\tt
sendDataTo}. The first function takes a bundle object, performs sanity checks
and passes it to the router as incoming bundle. The second function takes the
payload, the destination EID, and optionally the source EID and creates a new
bundle object which is passed to the router.

For either function RDTN expands the source EID if it is not a valid URI. In
that case the given string is taken as an application tag which is suffixed to
the local EID. For example, when the local EID configured for RDTN is {\tt
dtn://test.dtn/} and a sending function is called with the source EID {\tt app},
then RDTN expands this identifier, so that the source EID of the new bundle is
{\tt dtn://test.dtn/app}. If the source EID parameter is empty, the local EID is
used to fill this field.

\subsection{Receiving Bundles}\label{sec.receiving}

An application has to {\em register} if it wants to receive bundles using the
{\tt register}-function. A registration is always bound to an EID. Applications 
can register for any EID independent of the local EID of the RDTN instance.

When a registration is added, RDTN passes all stored bundles matching the
registered EID to the application. After that, whenever RDTN receives a bundle 
matching the registered EID, the application gets the bundle until the 
registration is cancelled.

The call to {\tt register} is not followed directly by a response. The
application gets the bundles asynchronously. For applications written in Ruby,
the {\tt register}-function takes a block which is called for each received
bundle matching the registration.

When the application has received a bundle, it can generate an {\em application
acknowledgement} that is sent to the bundle's source. Such a message can be
generated using the {\tt applicationAck}-function.

\subsection{Interactive Mode}\label{sec.interactive}

While the default setup is to have the RDTN daemon and the applications run in
different processes, having both run in one process is helpful for testing and
debugging purposes. This is especially true, for interactively working with
RDTN. 

RDTN can be used interactively simply by loading the daemon as a module in
Ruby's interactive interpreter irb. {\tt irb -r lib/daemon.rb}

{\bf TBD: installation for RDTN so we get the paths right.}

An interactive RDTN session with irb automatically creates a Daemon object (an
instance of the class {\tt RdtnDaemon::Daemon}) which is assigned to the global
variable {\tt \$daemon}. This object can be used to send bundles (see section
\ref{sec.sending}, register for receiving bundles (see section
\ref{sec.receiving}), and to manipulate the daemon's configuration. No
configuration file is read when an interactive session is started. Instead the
daemon uses a default configuration without convergence layer and with a static
table based router. When a configuration file is needed, it can be loaded
using the {\tt parseConfigFile}-function. Routing configuration, convergence
layers, storage settings, and discovery modules (see section
\ref{sec.contact-mngt}) can be manipulated through the daemon object.

An example session which sends a bundle to itself could look like this:

\begin{verbatim}
> $daemon.register("dtn://rdtn/test") do |b|
*   puts "Received: #{b.payload} from #{b.srcEid}"
> end
=> [#<Proc:0xb7ba44a8@./lib/routetab.rb:37>]
> $daemon.sendDataTo("Hello, rdtn!", "dtn://rdtn/test",
*                    "dtn://rdtn/sender")
Received: Hello, rdtn! from dtn://rdtn/sender
\end{verbatim}

\subsection{The RDTN Client Protocol}\label{sec.client-protocol}

The RDTN client protocol allows applications running in processes separate from
the daemon to use the application interface. We anticipate that most
applications will run on the same device as the daemon, but the interface can be
used over a network, as it is based on TCP. The application interface is,
however, not intended to be used over a challenged network, so the connections
are expected to be stable and have short round-trip times. A use case for
distributed applications is a vehicular network, where an RDTN router is
installed in a vehicle to handle communications in the challenged environment
while the vehicle is moving. Other devices in the vehicle can run the
applications which use the RDTN daemon.

The client protocol is designed with three goals in mind: language-independence,
straight-forward integration with the Ruby code of the daemon, and efficient
handling of large binary payloads. 

{\bf TBD: extend this part, when the YAML stuff is implemented.}

The protocol has two categories of messages: Requests sent from client to the
daemon that directly responds, and asynchronous messages from the daemon to the
client. Asynchronous messages are solicited by a request from the client. The
functions of the application interface described in sections \ref{sec.sending}
and \ref{sec.receiving} are mapped to requests of the client protocol. In
addition to these, the protocol provides requests to get a list of bundles from
the persistent storage and to delete bundles.

All messages of the client protocol have a message type, a message id, and a
number of type-specific arguments. The message type of requests maps to the name
of the corresponding function in Ruby's symbol notation (i.e. request
corresponding to the function {\tt sendDataTo} has the type {\tt :sendDataTo}).
A message from the daemon which contains bundles has the type {\tt
:bundles}. And responses for requests that solicit asynchronous messages get
responses with the types {\tt :ok} and {\tt :error} on success or failure
respectively.

The message id is created by the client and must be unique for all requests
during a connection with the daemon. Responses have the same message id as the
corresponding request. An asynchronous message has the same message id as the
request that solicited it. E.g., the client registers for receiving bundles with
a message type {\tt :register} and the message id {\tt 1}. The daemon responds
with an {\tt :ok}-message which also has the message id {\tt 1}. When a bundle
matching the registration arrives, the daemon sends a {\tt :bundles}-message,
again with message id {\tt 1}.

The arguments of the request are the same as for the functions. The {\tt
:ok}-message has no arguments, {\tt :error} is accompanied by an error message,
and messages of type {\tt :bundles} have a list of bundles as their arguments,
even when there is only a single bundle.

{\bf TBD: encoding}.

The daemon-side of this protocol is implemented in the classes {\tt
AppIF::AppInterface} and {\tt AppIF::AppLink}. The objects of the former class
wait for incoming connections and create objects of the latter class when a
client connects. {\tt AppIF::AppLink} instances handle the communication with
one client each, relaying requests between the client to the daemon's internal
interface.

\section{Configuration}\label{sec.config}

\section{Simulations}\label{sec.sim}

\section{Extending RDTN}\label{sec.extending}

\section{Conclusions}\label{sec.conclusions}

%\bibliography{bibtex/rn}

\end{document}
